% A1 Portrait Poster Template (beamerposter)
\documentclass[final]{beamer}

% Poster package
\usepackage[size=a1,orientation=portrait,scale=1.2]{beamerposter}

% Theme
\usetheme{default}

% Logo positioning
\usepackage{graphicx}
\usepackage{qrcode}
\usepackage{textpos}
\usepackage{graphicx}
\usepackage{tikz}
\usetikzlibrary{positioning,calc}

\setlength{\TPHorizModule}{1cm}
\setlength{\TPVertModule}{1cm}

\setlength{\topmargin}{-2cm}
\setlength{\headsep}{0pt}
\setlength{\headheight}{0pt}


% Colors
\definecolor{darkblue}{RGB}{20,20,80}
\definecolor{lightgray}{RGB}{240,240,240}

\setbeamercolor{block title}{fg=white,bg=darkblue}
\setbeamercolor{block body}{fg=black,bg=lightgray}
\setbeamerfont{block title}{size=\Large}
\setbeamerfont{block body}{size=\large}
\setbeamerfont{title}{size=\Huge}


\title{Impact of Class Imbalance Handling in Neural Networks}
\author{Hakan Körpe}
\institute{TU Dortmund -- AIML Course}

\begin{document}
\begin{frame}[t]{} % align content to top
\framesetup{width=\paperwidth} % allow full-width top block

% Remove default top spacing
\vspace*{-1.0cm}

% ======================
% TITLE BAR WITH QR CODE
% ======================
% ============================================
% TOP TITLE BAR WITH QR CODE (no textblock)
% ============================================
\begin{tikzpicture}[remember picture,overlay]
    % Title background bar (full width)
    \node[anchor=north west, inner sep=0pt] 
        at (current page.north west)
        (titlebar)
        {\begin{minipage}{\paperwidth}
            \begin{beamercolorbox}[wd=\paperwidth,ht=5cm,dp=1cm,center]{block title}
                {\fontsize{50}{56}\selectfont \textbf{Impact of Class Imbalance Handling in Neural Networks}}\\[0.4em]
                {\Large Hakan Körpe — TU Dortmund}
            \end{beamercolorbox}
        \end{minipage}};
\end{tikzpicture}

% QR CODE IN TOP-RIGHT (aligned inside the bar, not clipped)

\begin{tikzpicture}[remember picture,overlay]
    \node[anchor=north east, xshift=-1.2cm, yshift=-1.2cm] at (current page.north east)
        {\color{white}\qrcode[height=5em]{https://github.com/hakankorpe/AIML-class-imbalance}};
\end{tikzpicture}

\vspace*{3.5cm} % pushes your columns down below the title bar


\begin{columns}[t,totalwidth=\textwidth]

% LEFT COLUMN -----------------------------------------------------------
\begin{column}{0.32\textwidth}

\begin{block}{Abstract}
Class imbalance is a major challenge in machine learning. This poster compares
the effect of class weighting and SMOTE on both tabular (Breast Cancer) and
image-based (MNIST Binary) datasets. Metrics include F1, ROC-AUC, PR-AUC,
and confusion matrices. We evaluate three datasets to show how imbalance severity affects different modalities (tabular, digits, clothing images).
\end{block}

\begin{block}{Datasets}
\textbf{Breast Cancer (sklearn)}: binary classification with mild imbalance.\\
\textbf{MNIST Binary}: class ``0'' vs ``non-zero'', minority ratio 2\%.\\
\textbf{Fashion-MNIST Binary}: class ``T-shirt/top'' vs ``Other'', minority ratio 2\%.
\end{block}

\begin{block}{Methods}
\begin{itemize}
    \item Baseline neural network classifier
    \item Class weighting using sklearn balanced weights
    \item SMOTE oversampling for synthetic data generation
    \item Evaluation: Confusion Matrix, ROC-AUC, PR-AUC
\end{itemize}
\end{block}

\begin{block}{Breast Cancer Results}
\centering
% --- Class Weights ---
\includegraphics[width=\linewidth]{images/bc_confusion_weights.png}\\[0.5em]
\small\textbf{Class Weights – Confusion matrix.}
Most negatives are classified correctly, but only about half of the positives are detected.

\vspace{1.8em}

\includegraphics[width=\linewidth]{images/bc_roc_pr_weights.png}\\[0.5em]
\small\textbf{Class Weights – ROC \& PR.}
ROC-AUC is high, but PR-AUC reveals limited precision due to extreme class imbalance.

\vspace{2.0em}

\end{block}
\end{column}

% MIDDLE COLUMN ---------------------------------------------------------
\begin{column}{0.32\textwidth}

\begin{block}{Breast Cancer SMOTE Results}
\centering


% --- SMOTE ---
\includegraphics[width=\linewidth]{images/bc_confusion_smote.png}\\[0.5em]
\small\textbf{SMOTE – Confusion matrix.}
SMOTE increases minority recall at the cost of a few additional false positives.

\vspace{1.8em}

\includegraphics[width=\linewidth]{images/bc_roc_pr_smote.png}\\[0.5em]
\small\textbf{SMOTE – ROC \& PR.}
SMOTE improves PR-AUC significantly, showing better precision–recall trade-off.

\normalsize
\end{block}

\begin{block}{MNIST Binary Results}
\centering

\includegraphics[width=\linewidth]{images/mnist_confusion.png}\\[0.5em]
\small\textbf{MNIST Binary – Confusion matrix.}
Both classes are recognized very well: almost all non-zero digits and most zeros are classified correctly.\\[1.0em]

\vspace{1.8em}

\includegraphics[width=\linewidth]{images/mnist_roc_pr.png}\\[0.5em]
\small\textbf{MNIST Binary – ROC \& PR.}
ROC-AUC is close to 1.0, and the PR curve shows high precision even at high recall, meaning the classifier is very confident on this task.

\end{block}


\end{column}

% RIGHT COLUMN ----------------------------------------------------------
\begin{column}{0.32\textwidth}

\begin{block}{Fashion-MNIST Binary Results}
\centering



% --- Fashion MNIST ---
\includegraphics[width=\linewidth]{images/fmnist_confusion.png}\\[0.5em]
\small\textbf{Fashion-MNIST – Confusion matrix.}\\
T-shirt/top detection is harder: more overlap with other clothing categories, leading to lower minority recall.\\[1.0em]


\vspace{1.8em}


\includegraphics[width=\linewidth]{images/fmnist_roc_pr.png}\\[0.5em]
\small\textbf{Fashion-MNIST – ROC \& PR.}\\
ROC is strong, but the PR-AUC drops noticeably — a sign that class imbalance is much more challenging here.

\normalsize
\end{block}


\begin{block}{Results Table}
\begin{tabular}{lccc}
\textbf{Method} & \textbf{F1} & \textbf{ROC-AUC} & \textbf{PR-AUC} \\ \hline
Class Weights & 0.87 & 0.92 & 0.56 \\
SMOTE         & 0.90 & 0.96 & 0.79 \\
MNIST CW      & 1.00 & 0.99 & 0.73 \\
FMNIST CW     & 0.86 & 0.96 & 0.10 \\
\end{tabular}
\end{block}

\vspace{1.0em}

\begin{block}{Conclusion}
\begin{itemize}
    \item SMOTE significantly improves minority recall.
    \item Class weighting is stable and simple for tabular data.
    \item MNIST shows strong separability even under imbalance.
    \item FMNIST is significantly harder, visible from low PR-AUC.
    \item ROC-AUC is misleadingly high under extreme imbalance (important insight!).
\end{itemize}
\end{block}


\end{column}

\end{columns}

% ======================
% BOTTOM LOGO BAR
% ======================
% ======================
% BOTTOM BAR (same color as title)
% ======================
\begin{tikzpicture}[remember picture,overlay]

    % --- Bottom dark blue bar ---
    \node[anchor=south west, inner sep=0pt] at (current page.south west)
        {\begin{minipage}{\paperwidth}
            \begin{beamercolorbox}[wd=\paperwidth,ht=3.5cm,dp=0.5cm]{block title}
            \end{beamercolorbox}
        \end{minipage}};

    % --- TU Dortmund logo, forced to white ---
    \node[anchor=south west, xshift=1.2cm, yshift=0.8cm] at (current page.south west)
        {\begingroup
            \color{white}%
            \includegraphics[height=2.5cm]{images/Technische_Universität_Dortmund_Logo.png}%
        \endgroup};

\end{tikzpicture}



\end{frame}
\end{document}
