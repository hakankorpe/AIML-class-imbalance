% A1 Portrait Poster Template (beamerposter)
\documentclass[final]{beamer}

% Poster package
\usepackage[size=a1,orientation=portrait,scale=1.2]{beamerposter}

% Theme
\usetheme{default}

% Logo positioning
\usepackage{graphicx}
\usepackage{qrcode}
\usepackage{textpos}
\usepackage{graphicx}
\usepackage{tikz}
\usetikzlibrary{positioning,calc}

\setlength{\TPHorizModule}{1cm}
\setlength{\TPVertModule}{1cm}

\setlength{\topmargin}{-2cm}
\setlength{\headsep}{0pt}
\setlength{\headheight}{0pt}


% Colors
\definecolor{darkblue}{RGB}{20,20,80}
\definecolor{lightgray}{RGB}{240,240,240}

\setbeamercolor{block title}{fg=white,bg=darkblue}
\setbeamercolor{block body}{fg=black,bg=lightgray}
\setbeamerfont{block title}{size=\Large}
\setbeamerfont{block body}{size=\large}
\setbeamerfont{title}{size=\Huge}


\title{Impact of Class Imbalance Handling in Neural Networks}
\author{Hakan Körpe}
\institute{TU Dortmund -- AIML Course}

\begin{document}
\begin{frame}[t]{} % align content to top
\framesetup{width=\paperwidth} % allow full-width top block

% Remove default top spacing
\vspace*{-1.0cm}

% ======================
% TITLE BAR WITH QR CODE
% ======================
% ============================================
% TOP TITLE BAR WITH QR CODE (no textblock)
% ============================================
\begin{tikzpicture}[remember picture,overlay]
    % Title background bar (full width)
    \node[anchor=north west, inner sep=0pt] 
        at (current page.north west)
        (titlebar)
        {\begin{minipage}{\paperwidth}
            \begin{beamercolorbox}[wd=\paperwidth,ht=5cm,dp=1cm,center]{block title}
                {\fontsize{50}{56}\selectfont \textbf{Impact of Class Imbalance Handling in Neural Networks}}\\[0.4em]
                {\Large Hakan Körpe — TU Dortmund}
            \end{beamercolorbox}
        \end{minipage}};
\end{tikzpicture}

% QR CODE IN TOP-RIGHT (aligned inside the bar, not clipped)

\begin{tikzpicture}[remember picture,overlay]
    \node[anchor=north east, xshift=-1.2cm, yshift=-1.2cm] at (current page.north east)
        {\color{white}\qrcode[height=5em]{https://github.com/hakankorpe/AIML-class-imbalance}};
\end{tikzpicture}

\vspace*{3.5cm} % pushes your columns down below the title bar


\begin{columns}[t,totalwidth=\textwidth]

% LEFT COLUMN -----------------------------------------------------------
\begin{column}{0.32\textwidth}

\begin{block}{Abstract}
Class imbalance is a major challenge in machine learning. This poster compares
the effect of class weighting and SMOTE on both tabular (Breast Cancer) and
image-based (MNIST Binary) datasets. Metrics include F1, ROC-AUC, PR-AUC,
and confusion matrices. We evaluate three datasets to show how imbalance severity affects different modalities (tabular, digits, clothing images).
\end{block}

\begin{block}{Datasets}
\textbf{Breast Cancer (sklearn)}: binary classification with mild imbalance.\\
\textbf{MNIST Binary}: class ``0'' vs ``non-zero'', minority ratio 2\%.\\
\textbf{Fashion-MNIST Binary}: class ``T-shirt/top'' vs ``Other'', minority ratio 2\%.
\end{block}

\begin{block}{Methods}
\begin{itemize}
    \item Baseline neural network classifier
    \item Class weighting using sklearn balanced weights
    \item SMOTE oversampling for synthetic data generation
    \item Evaluation: Confusion Matrix, ROC-AUC, PR-AUC
\end{itemize}
\end{block}

\begin{block}{Why Accuracy Fails Under Imbalance}
With only 2\% minority samples, always predicting the majority gives 98\% accuracy. 
ROC-AUC also remains high even for poor minority recall.
PR-AUC is needed to evaluate minority detection.
\end{block}

\begin{block}{Breast Cancer Results}
\centering
% --- Class Weights ---
\includegraphics[width=\linewidth]{images/bc_confusion_weights.png}\\[0.5em]
\small\textbf{Class Weights – Confusion matrix.}
Class weights help the model detect positives better, but recall remains unstable due to the extremely small number of minority samples.

\vspace{1.8em}

\includegraphics[width=\linewidth]{images/bc_roc_pr_weights.png}\\[0.5em]
\small\textbf{Class Weights – ROC \& PR.}
ROC-AUC is high, but PR-AUC reveals limited precision due to extreme class imbalance.

\vspace{2.0em}

\end{block}
\end{column}

% MIDDLE COLUMN ---------------------------------------------------------
\begin{column}{0.32\textwidth}

\begin{block}{Breast Cancer SMOTE Results}
\centering


% --- SMOTE ---
\includegraphics[width=\linewidth]{images/bc_confusion_smote.png}\\[0.5em]
\small\textbf{SMOTE – Confusion matrix.}
SMOTE increases minority recall at the cost of a few additional false positives.

\vspace{1.8em}

\includegraphics[width=\linewidth]{images/bc_roc_pr_smote.png}\\[0.5em]
\small\textbf{SMOTE – ROC \& PR.}
SMOTE improves PR-AUC significantly, showing better precision–recall trade-off.

\normalsize
\end{block}

\begin{block}{MNIST Binary Results}
\centering

\includegraphics[width=\linewidth]{images/mnist_confusion.png}\\[0.5em]
\small\textbf{MNIST Binary – Confusion matrix.}
Both classes are recognized very well: almost all non-zero digits and most zeros are classified correctly.\\[1.0em]

\vspace{1.8em}

\includegraphics[width=\linewidth]{images/mnist_roc_pr.png}\\[0.5em]
\small\textbf{MNIST Binary – ROC \& PR.}
ROC-AUC is close to 1.0, and the PR curve shows high precision even at high recall, meaning the classifier is very confident on this task.

\end{block}

\begin{block}{Results Table}
\centering
\begin{tabular}{lccc}
\textbf{Method} & \textbf{F1} & \textbf{ROC-AUC} & \textbf{PR-AUC} \\ \hline
Breast Cancer CW & 1.00 & 1.00 & 1.00 \\
Breast Cancer Smote & 0.62 & 0.94 & 0.64 \\
MNIST CW & 0.53 & 0.9988 & 0.807 \\
FMNIST CW & 0.07 & 0.9846 & 0.125 \\
\end{tabular}
\end{block}




\end{column}

% RIGHT COLUMN ----------------------------------------------------------
\begin{column}{0.32\textwidth}

\begin{block}{Digit Difficulty Under Class Imbalance}
\small
Different MNIST digits suffer from imbalance in different ways:

\begin{itemize}
    \item \textbf{Digits 0 and 1} remain easy to detect even when rare  
          (PR-AUC: 0.88–0.95).  
          Their shapes are simple and distinct → low confusion.
    
    \item \textbf{Digits 8 and 9} collapse under imbalance  
          (PR-AUC: 0.33–0.36).  
          They are visually similar to multiple other digits  
          (8 ↔ 3, 0, 9; 9 ↔ 4, 7).
    
    \item \textbf{Accuracy and ROC-AUC stay high}, but do not reflect the real difficulty.
    
    \item \textbf{PR-AUC reveals true minority-class performance}, showing large drops for complex digits.
\end{itemize}

\begin{center}
\includegraphics[width=\linewidth]{images/mnist_pr_auc_digits.png}\\
\small{PR-AUC comparison for minority digits 0, 1, 8, and 9.}
\end{center}

\end{block}


\begin{block}{Fashion-MNIST Binary Results}
\centering

 

% --- Fashion MNIST ---
\includegraphics[width=\linewidth]{images/fmnist_confusion.png}\\[0.5em]
\small\textbf{Fashion-MNIST – Confusion matrix.}\\
T-shirt/top detection is harder: more overlap with other clothing categories, leading to lower minority recall.\\[1.0em]

\includegraphics[width=\linewidth]{images/fmnist_roc_pr.png}\\[0.5em]
\small\textbf{Fashion-MNIST – ROC \& PR.}\\
ROC is strong, but the PR-AUC drops noticeably, a sign that class imbalance is much more challenging here.

\normalsize
\end{block}


\begin{block}{Conclusion}
\begin{itemize}
    \item Class weighting performs extremely well on Breast Cancer data. SMOTE increases recall but may reduce precision, lowering F1 and PR-AUC.
    \item MNIST remains highly separable even under extreme imbalance (2\% minority).
    \item Fashion-MNIST is far more difficult: PR-AUC collapses despite a high ROC-AUC.
    \item ROC-AUC can be misleading under imbalance; PR-AUC is the reliable metric.
    \item Digit-wise MNIST imbalance shows large difficulty differences: 1 is easy, while 8 and 9 are much harder.
\end{itemize}
\end{block}



\end{column}

\end{columns}

% ======================
% BOTTOM LOGO BAR
% ======================
% ======================
% BOTTOM BAR (same color as title)
% ======================
\begin{tikzpicture}[remember picture,overlay]

    % --- Bottom dark blue bar ---
    \node[anchor=south west, inner sep=0pt] at (current page.south west)
        {\begin{minipage}{\paperwidth}
            \begin{beamercolorbox}[wd=\paperwidth,ht=3.5cm,dp=0.5cm]{block title}
            \end{beamercolorbox}
        \end{minipage}};

    % --- TU Dortmund logo, forced to white ---
    \node[anchor=south west, xshift=1.2cm, yshift=0.8cm] at (current page.south west)
        {\begingroup
            \color{white}%
            \includegraphics[height=2.5cm]{images/tu.png}%
        \endgroup};

\end{tikzpicture}



\end{frame}
\end{document}
